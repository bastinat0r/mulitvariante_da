% erhebungsmethoden und foo
% 1 UV: Demographie
% 2 UV: Mobilitätsverhalten
% 3 AV: Ansichten zur Mobilität

\section{Theoretische Vorüberlegung}

\subsection{Begriffe}
\subsubsection*{Mobilität}

Mobilität ist die Möglichkeit zum Wechsel von Orten, mit der auch Symbolische und Soziale Aspekte verknüpft sind Verkehr lässt sich als realisierte Mobilität betrachten \parencite[6]{fh7}.
Bestimmend für den Möglichkeitsraum der Mobilität sind die Verkehrsmittel die wir benutzen – also die verfügbaretechnischen Mittel die uns von A nach B befördern und unser Mobilitätsverhalten prägen.
Unterscheiden kann man zwischen Physicher- und Sozialer Mobilität, also der Möglichkeit sich zwischen Verschiedenen Sozialen Räumen zu bewegen \parencite[67 f.]{schulz}.

\subsubsection*{Urbanität}
Urbanität findet sich sowohl im Sozialen- als auch im Räumlichen gefüge der Gesellschaft wieder.
Entscheidend für die Mobilität ist wohl, das in Urbanen Umgebungen wesentlich kürzere Wege bewältigt werden müssen als außerhalb der Städte.
Im Gegenzug dazu ist der Verkehr in Städten wesentlich dichter und die Auslastung der Verkehrsmittel ist wesentlich höher als in ländlichen Regionen.
\todo{urban = in städtischer umgebung + quelle}

\subsection{Makroebene}
Die Theorien der Makroebene beschäftigen sich vor allem mit den Auswirkungen von Mobilität auf die Gesellschaft \parencite[6 ff.]{widmer}.
Ein Beispiel dafür ist die Beschleunigung des Lebenstempos \parencite{rosa,rosa1} oder die Flexibilisierung der Gesellschaft \parencite{sennett}.

Eine Entwicklung der letzten Jahre ist, dass die Verbindung zwischen Physicher- und Sozialer Mobilität zunehmend abnimmt, da Kommunikation auch über große Distanzen stattfinden kann (\enquote{virtual mobility}).
Außerdem wird Mobilität nicht mehr nur als Möglichkeit, sondern als Realität wahrgenommen \parencite[17 f.]{widmer}.

\subsection{Mikroebene}
Mobilität hat nicht zuletzt auch Auswirkungen auf jeden einzelnen Akteur und ist sowohl in seiner Eigenschaft als Möglichkeitsraum (vgl \cite{riessman}) als auch in der Symbolischen Bedeutung, die unseren Verkehrsmitteln innewohnt, identitätsstiftend.
Insofern ist auch davon auszugehen, das unser Mobilitätsverhalten unsere Ansichten zur Mobilität beeinflussen.


\subsection{Mesoebene}

Die meisten Theorien zur Mobilität verknüpfen Handlungsmotivationen in der Mikroebene mit Auswirkungen sowohl in Mikro- als auch Makroebene, diese Theorien lassen sich in zwei Gruppen einteilen: \enquote{Rational choice theorie} (Theorie der Rationalen Entscheidung) und \enquote{cultural or life-style theories} \parencite{widmer}.

\subsubsection*{Rational Choice}
Kern der rational choice theorie ist die Annahme, das alle Akteure rationale Entscheidungen im Sinne einer Kosten-Nutzen-Maximierung für sich selbst treffen \parencite[][19]{wikiRationalChoice, widmer}.
Das einfachste Modell dieser Kathegorie ist vermutlich der Homo Oeconomicus.
Neuere RC-Theorien gehen davon aus, dass die getroffenen Entscheidungen unter bestimmten Beschränkungen (bounded rationality) und ohne vollständiges Wissen über deren Auswirkungen getroffen werden.
Die Theorien haben das Problem, dass die subjektiven Kosten und der subjektive Nutzen einer Handlung durch einen Beobachter praktisch nicht abschätzbar sind und altruisitsche Handlungsmotive ausgeschlossen werden.

Da die kosten für die meisten Handlungen im bereich der Mobilität allerdinsgs relativ hoch sind spielt die Abschätzung von Kosten und Nutzen beim Treffen einer Entscheidung relativ sicher eine Rolle, und die Motivation eine Informierte Entscheidung zu treffen ist relativ hoch \parencite[21]{widmer}\footnote{In der Qulle beziehen sich die beiden Autoren explizit auf Arbeitsmobilität, ich denke aber die Aussage lässt sich durchaus verallgemeinern.}.

\subsubsection*{Mobilitätskultur}
\begin{quote}
	\dots aspects of mobility culture seem clearly detectable. In order to become mobile it is
important to know about traffic signs and speed limits, to know about train schedules and how
to research them, to know how much luggage is allowed on an airplane etc. \emph{\parencite[24]{widmer}}
\end{quote}

Ein anderer Ansatz als der der Rationlen Entscheidungen ist der Bezug auf eine \emph{Mobilitätskultur} \parencite[22 f.][6 ff.]{widmer, fh7}.
Das Mobilitätsverhalten eines Einzelnen ist damit Ergebnis eines Sozialisierungsprozesses.
Dabei können nicht nur Unterschiede in der Mobilitätskultur aus verschiedenen Kulturkreisen verglichen werden, sondern durchaus auch innerhalb eines Gebietes mit sehr ähnlichen Kulturen z.B. innerhalb Europas – so lassen sich zwischen Frankfurt und Zürich deutliche Unterschiede in der Mobilitätskultur feststellen \parencite{fh7}.

Auch verschiedene Randbedingungen bei ähnlicher Kultur kann zu unterschieden führen \parencite[24]{widmer}.
Dabei sind auch Subkulturen und die Mischung an verschiedenen Subkulturen an denen ein Einzelner teilnimmt entscheidend für das Mobilitätsverhalten.
So sind die Einstellungen zum ÖPNV in verschiedenen Städten nicht nur durch rationales Denken, sondern auch in der Kultur der jeweiligen Stadt verankert.

Diese Einstellungen sind Bestandteile eines Life-Style \parencite[24 f.]{widmer} der wiederum als Identitätsangebot fungiert.
Die kulturellen Gegebenheiten schlagen sich auch in Rollenbildern und Erwartungen an diese wieder.
Den bestehenden Leitbildern wird teilweise auch völlig unreflektiert gefolgt \parencite[27]{widmer}.
Das theoretische Fundament dieser Ansätze findet sich unter anderem im Symbolischen Interaktionismus \parencite[27][14 ff.]{widmer, schulz}.
