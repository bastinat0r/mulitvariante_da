% erhebungsmethoden und foo
% 1 UV: Demographie
% 2 UV: Mobilitätsverhalten
% 3 AV: Ansichten zur Mobilität

\section{Theoretische Vorüberlegung}

Für die Auswertung der Umfragen verwenden wir drei Blöcke von Variablen:
\begin{description}
	\item[UV - Demographie] Demographische Faktoren, wie Alter, sozialer Status und Geschlecht, Nationalität
	\item[UV - Mobilitätsverhalten] Mit welchen Mitteln bewegen sich die Probanden im Straßenverkehr
	\item[AV - Ansichten zur Verkehrspolitik] Welche Ziele sollen von der Politik im Verkehrsraum verfolgt werden, welche Probleme sehen die Probanden
\end{description}
Das die Ansichten der Befragten Menschen von demographischen Faktoren abhängen sollte offensichtlich sein.
Dagegen ist nicht völlig offensichtlich, ob das Mobilitätsverhalten der Probanden die Ansichten zur Verkehrspolitik ändern, oder ob eine andere Einstellung der Probanden dafür sorgt, dass andere Verkehrsmittel genutzt werden – Realistisch betrachtet dürfte beides der Fall sein.
Wir beschränken uns allerdings darauf, die Auswirkungen des Mobilititätsverhaltens auf die Ansichten der Probanden zu untersuchen.
So wird ein Fahrradfahrer beispielsweise den Straßenverkehr anders wahrnehmen und unter Umständen auch andere Ansichten zur Verkehrspolitik entwickeln.



