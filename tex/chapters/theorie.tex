% erhebungsmethoden und foo
% 1 UV: Demographie
% 2 UV: Mobilitätsverhalten
% 3 AV: Ansichten zur Mobilität

\section{Theoretische Vorüberlegung}

\subsection{Begriffe}
\subsubsection*{Mobilität}

Mobilität ist die Möglichkeit zum Wechsel von Orten, mit der auch Symbolische und Soziale Aspekte verknüpft sind Verkehr lässt sich als realisierte Mobilität betrachten \parencite[6]{fh7}.
Bestimmend für den Möglichkeitsraum der Mobilität sind die Verkehrsmittel die wir benutzen – also die verfügbaretechnischen Mittel die uns von A nach B befördern und unser Mobilitätsverhalten prägen.
Unterscheiden kann man zwischen Physicher- und Sozialer Mobilität, also der Möglichkeit sich zwischen Verschiedenen Sozialen Räumen zu bewegen \parencite[67 f.]{schulz}.

\subsubsection*{Urbanität}
Urbanität findet sich sowohl im Sozialen- als auch im Räumlichen gefüge der Gesellschaft wieder.
Entscheidend für die Mobilität ist wohl, das in Urbanen Umgebungen wesentlich kürzere Wege bewältigt werden müssen als außerhalb der Städte.
Im Gegenzug dazu ist der Verkehr in Städten wesentlich dichter und die Auslastung der Verkehrsmittel ist wesentlich höher als in ländlichen Regionen.
\todo{urban = in städtischer umgebung + quelle}

\subsection{Makroebene}
Die Theorien der Makroebene beschäftigen sich vor allem mit den Auswirkungen von Mobilität auf die Gesellschaft \parencite[6 ff.]{widmer}.
Ein Beispiel dafür ist die Beschleunigung des Lebenstempos \parencite{rosa,rosa1} oder die Flexibilisierung der Gesellschaft \parencite{sennett}.
Eine Entwicklung der letzten Jahre ist, dass die Verbindung zwischen Physicher- und Sozialer Mobilität zunehmend abnimmt, da Kommunikation auch über große Distanzen stattfinden kann (\enquote{virtual mobility}).
Außerdem wird Mobilität nicht mehr nur als Möglichkeit, sondern als Realität wahrgenommen \parencite[17 f.]{widmer}.

\subsection{Mikroebene}
Mobilität hat nicht zuletzt auch Auswirkungen auf jeden einzelnen Akteur und ist sowohl in seiner Eigenschaft als Möglichkeitsraum (vgl \cite{riessman}) als auch in der Symbolischen Bedeutung, die unseren Verkehrsmitteln innewohnt, identitätsstiftend.
Insofern ist auch davon auszugehen, das unser Mobilitätsverhalten unsere Ansichten zur Mobilität beeinflussen.

\subsection{Mesoebene}


% mobilität als identitätsstiftendes kriterium

