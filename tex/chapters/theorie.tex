% erhebungsmethoden und foo
% 1 UV: Demographie
% 2 UV: Mobilitätsverhalten
% 3 AV: Ansichten zur Mobilität

\section{Theoretische Vorüberlegung}

\subsection{Begriffe}
\subsubsection*{Mobilität}

Mobilität ist die Möglichkeit zum Wechsel von Orten, mit der auch Symbolische und Soziale Aspekte verknüpft sind Verkehr lässt sich als realisierte Mobilität betrachten \parencite[6]{fh7}.
Bestimmend für den Möglichkeitsraum der Mobilität sind die Verkehrsmittel die wir benutzen – also die verfügbaretechnischen Mittel die uns von A nach B befördern und unser Mobilitätsverhalten prägen.

\subsubsection*{Urbanität}
Urbanität findet sich sowohl im Sozialen- als auch im Räumlichen gefüge der Gesellschaft wieder.
Entscheidend für die Mobilität ist wohl, das in Urbanen Umgebungen wesentlich kürzere Wege bewältigt werden müssen als außerhalb der Städte.
Im Gegenzug dazu ist der Verkehr in Städten wesentlich dichter und die Auslastung der Verkehrsmittel ist wesentlich höher als auf dem Land.
\todo{urban = in städtischer umgebung + quelle}

\subsection{Variablen}

Bei den Ansichten zur Verkehrspolitik gibt es vermutlich zwei Konkurrierende Ziele: Einerseits die Erhaltung und Erweiterung der eigenen Mobilität und andererseits die Förderung der Nachhaltigkeit der genutzten Verkehrsmittel.

Für die Auswertung der Umfragen verwenden wir drei Blöcke von Variablen:
\begin{description}
	\item[UV - Demographie] Demographische Faktoren, wie Alter, sozialer Status und Geschlecht, Nationalität
	\item[UV - Mobilitätsverhalten] Mit welchen Mitteln bewegen sich die Probanden im Straßenverkehr
	\item[AV - Ansichten zur Verkehrspolitik] Welche Ziele sollen von der Politik im Verkehrsraum verfolgt werden, welche Probleme sehen die Probanden
\end{description}
Das die Ansichten der Befragten Menschen von demographischen Faktoren abhängen sollte offensichtlich sein.
Dagegen ist nicht völlig offensichtlich, ob das Mobilitätsverhalten der Probanden die Ansichten zur Verkehrspolitik ändern, oder ob eine andere Einstellung der Probanden dafür sorgt, dass andere Verkehrsmittel genutzt werden – Realistisch betrachtet dürfte beides der Fall sein.
Wir beschränken uns allerdings darauf, die Auswirkungen des Mobilititätsverhaltens auf die Ansichten der Probanden zu untersuchen.
So wird ein Fahrradfahrer beispielsweise den Straßenverkehr anders wahrnehmen und unter Umständen auch andere Ansichten zur Verkehrspolitik entwickeln.
\nocite{schulz}
