\section{Mobilität im Urbanen Raum} % (fold)
	
	Mobilität ist ein wichtiger Faktor für alle Menschen in einer Modernen Gesellschaft.
	Wir möchten untersuchen, wie sich die Ansichten der Menschen in Städten zu ihrem Mobilitätsverhalten in Beziehung setzen lassen.

\begin{quote}
	Transport is a key enabler of social and economic development, and the transport sector
accounts for 9 million jobs across the EU. As 68\% of EU citizens live in urban areas 1 ,
urban transport is particularly important to future growth.
\emph{\parencite[2]{ebs406en}}
\end{quote}
	Die Popularität diverser neuer Technologien wie Elektromobilität oder Car-Sharing zeigt das dieses Thema durchaus Interessant werden wird.
	Abgesehen von der bloßen logistischen Komponente des Transportes von allerlei Dingen, ist auch die Frage \enquote{Wie komme ich von hier nach dort?} einem Wandel unterzogen – schon allein, weil jemand der Einkaufen geht das vielleicht heute bei Amazon tut und sich damit gar nicht mehr physisch bewegen muss.

	Besonders Interessant ist die Entwicklung der Mobilität in Urbanen Regionen, da dort nicht nur die meisten Menschen wohnen, sondern der Verkehr auch besonders Dicht und die Diversität an verschiedenen Verkehrsmitteln am größten ist.
	Schließlich sind durch die kürzeren Wege und den dichteren Verkehr in der Stadt ganz andere Verkehrsmittel nutzbar als auf dem Land – andererseits treten hier auch die größten Probleme zutage: Die Straßen sind verstopft und auch die Umweltbelastung durch den Verkehr fällt hier zuerst ins Auge.

	Im Gegensatz zu den meisten anderen Studien die Ausgehend vom Milleu und sozialer Verortung das Mobilitätsverhalten untersuchen \parencite{widmer}, wollen wir untersuchen wie sich das Mobilitätsverhalten und die daraus resultierenden unterschiedlichen Perspektiven auf die Ansichten zur Verkehrspolitik auswirken.
	Änderungen der Mobilitätskultur funktionieren am besten durch Positives-Feedback und einen stetige Verkehrspolitik \parencite{fh7}, wenn man die Ansichten der Einzelnen Mobilitätsgruppen voneinander trennen kann, könnte man Erkennen, wann so ein positives Feedback auftreten könnte – zum Beispiel, weil in einer Stadt mit einem höheren Anteil an Fahrradfahrern (und dem Ausbau der entsprechenden Infrastruktur), das Fahrradfahren attraktiver wird – woraufhin mehr Leute aufs Fahrrad umsteigen.

	Einige Studien zum Thema Mobilität vergleichen auch einzlne Städte oder Regionen miteinander \parencite{fh7, widmer} um bei gleicher Kultur Regionale unterschiede zu erkennen, oder vergleichen die Verkehrskonzepte verschiedener Nationalitäten um kulturelle Unterschiede auszumachen.
	Wir jedoch versuchen durch eine Analyse der verschiedenen europäischen Stadtbewohner sowohl regionale- als auch kulturelle Unterschiede auszublenden.

	In Europa fahren jeden Tag mehr Menschen mit dem Auto, als öffentliche Verkehrsmittel und Fahrräder zusammen – und das meistens allein.
	Das Auto ist auch Teil unserer Kultur und nicht selten ein Symbol für unseren sozialen Status.
	Doch auch ander Verkehrsmittel funktionieren als Statussymbol: Eine Bahncard100 oder ein besonderes Fahrrad machen ebenfalls eine Statusaussage – die vermutlich aber Aufgrund der vorherrschenden Mobilitätskultur als weniger bedeutend gewertet werden.
	Und es ist sicher auch Interessant, wie das eigene Mobilitätsverhalten die Bewertung dieser verändert.
	Die Frage ist, ob alternative Verkehrsmittel trotzdem Identitätsangebote bieten können – seine Identität im ÖPNV zu finden ist sicher schwieriger als sich über sein Auto Identität zu konstruieren.

	Letztendlich ist fraglich, ob sich das momentane Moblitätskonzept in der heutigen Form vortsetzen lässt – sowohl aus Gründen der Nachhaltigkeit und des Umweltschutzes, als auch weil die Technologien die wir heute für unsere Fortbewegung einsetzen bald veraltet sind.

	

% subsection Mobilität im Urbanen Raum (end)
