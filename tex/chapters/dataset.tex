\section{Datensatz}
% Metadata für unseren Datensatz, 3-4 seiten
% mit univariaten verteilungen als tabelle

\subsection{Erhebung}

\subsection{Metadaten} % (fold)
\label{sub:Metadaten}

Für den Datensatz wurden Daten aus 28 Verschiedenen Ländern der EU erhoben.
\begin{floatingtable}[r] {
		\begin{tabular}{l | l | l}
			von & bis & Land \\ \hline
			25.05.2013 & 09.06.2013 & Belgium \\
			25.05.2013 & 02.06.2013 & Bulgaria \\
			24.05.2013 & 06.06.2013 & Czech Republic \\
			25.05.2013 & 09.06.2013 & Denmark \\
			24.05.2013 & 09.06.2013 & Germany \\
			24.05.2013 & 09.06.2013 & Estonia \\
			25.05.2013 & 09.06.2013 & Ireland \\
			25.05.2013 & 08.06.2013 & Greece \\
			24.05.2013 & 09.06.2013 & Greece \\
			24.05.2013 & 08.06.2013 & France \\
			25.05.2013 & 07.06.2013 & Italy \\
			24.05.2013 & 09.06.2013 & Rep. of Cyprus \\
			25.05.2013 & 09.06.2013 & Latvia \\
			25.05.2013 & 09.06.2013 & Lithuania \\
			25.05.2013 & 09.06.2013 & Luxembourg \\
			25.05.2013 & 09.06.2013 & Hungary \\
			24.05.2013 & 09.06.2013 & Malta \\
			24.05.2013 & 09.06.2013 & Netherlands \\
			24.05.2013 & 09.06.2013 & Austria \\
			25.05.2013 & 09.06.2013 & Poland \\
			28.05.2013 & 09.06.2013 & Portugal \\
			25.05.2013 & 04.06.2013 & Romania \\
			25.05.2013 & 09.06.2013 & Slovenia \\
			25.05.2013 & 09.06.2013 & Slovakia \\
			25.05.2013 & 09.06.2013 & Finland \\
			25.05.2013 & 09.06.2013 & Sweden \\
			25.05.2013 & 09.06.2013 & United Kingdom \\
			25.05.2013 & 09.06.2013 & Croatia \\
		\end{tabular}}
	\caption{Erhebungszeiträume und Länder: \cite{eb794}}
\end{floatingtable}

In jedem Land wurden ca. 1000 Menschen in computergeleiteten face-to-face Interviewes befragt, in Deutschland wurden 1000 Menschen in West- und 500 in Ostdeutschland befragt. In Luxembourg aufgrund der geringen Bevölkerungszahl nur 600 und in Großbritannien 1300. Insgesamt wurden 27680 Menschen befragt \parencite{eceuropaeu}.

% subsection Metadaten (end)
