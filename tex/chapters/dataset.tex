\section{Datensatz}
% Metadata für unseren Datensatz, 3-4 seiten
% mit univariaten verteilungen als tabelle

\nocite{eb794}
\subsection{Erhebung}

Für den Datensatz wurden Daten aus 28 Verschiedenen Ländern der EU erhoben.
%\begin{floatingtable}[r] {
%		\begin{tabular}{l | l | l}
%			von & bis & Land \\ \hline
%			25.05.2013 & 09.06.2013 & Belgium \\
%			25.05.2013 & 02.06.2013 & Bulgaria \\
%			24.05.2013 & 06.06.2013 & Czech Republic \\
%			25.05.2013 & 09.06.2013 & Denmark \\
%			24.05.2013 & 09.06.2013 & Germany \\
%			24.05.2013 & 09.06.2013 & Estonia \\
%			25.05.2013 & 09.06.2013 & Ireland \\
%			25.05.2013 & 08.06.2013 & Greece \\
%			24.05.2013 & 09.06.2013 & Greece \\
%			24.05.2013 & 08.06.2013 & France \\
%			25.05.2013 & 07.06.2013 & Italy \\
%			24.05.2013 & 09.06.2013 & Rep. of Cyprus \\
%			25.05.2013 & 09.06.2013 & Latvia \\
%			25.05.2013 & 09.06.2013 & Lithuania \\
%			25.05.2013 & 09.06.2013 & Luxembourg \\
%			25.05.2013 & 09.06.2013 & Hungary \\
%			24.05.2013 & 09.06.2013 & Malta \\
%			24.05.2013 & 09.06.2013 & Netherlands \\
%			24.05.2013 & 09.06.2013 & Austria \\
%			25.05.2013 & 09.06.2013 & Poland \\
%			28.05.2013 & 09.06.2013 & Portugal \\
%			25.05.2013 & 04.06.2013 & Romania \\
%			25.05.2013 & 09.06.2013 & Slovenia \\
%			25.05.2013 & 09.06.2013 & Slovakia \\
%			25.05.2013 & 09.06.2013 & Finland \\
%			25.05.2013 & 09.06.2013 & Sweden \\
%			25.05.2013 & 09.06.2013 & United Kingdom \\
%			25.05.2013 & 09.06.2013 & Croatia \\
%		\end{tabular}}
%	\caption{Erhebungszeiträume und Länder: \cite{eb794}}
%\end{floatingtable}

In jedem Land wurden ca. 1000 Menschen in computergeleiteten face-to-face Interviewes befragt, in Deutschland wurden 1000 Menschen in West- und 500 in Ostdeutschland befragt. In Luxembourg aufgrund der geringen Bevölkerungszahl nur 600 und in Großbritannien 1300. Insgesamt wurden 27680 Menschen befragt \parencite{eceuropaeu}.

\subsection{Variablen}

Bei den Ansichten zur Verkehrspolitik gibt es vermutlich zwei Konkurrierende Ziele: Einerseits die Erhaltung und Erweiterung der eigenen Mobilität und andererseits die Förderung der Nachhaltigkeit der genutzten Verkehrsmittel.

Für die Auswertung der Umfragen verwenden wir drei Blöcke von Variablen:
\begin{description}
	\item[UV - Demographie] Demographische Faktoren, wie Alter, sozialer Status und Geschlecht, Nationalität
	\item[UV - Mobilitätsverhalten] Mit welchen Mitteln bewegen sich die Probanden im Straßenverkehr
	\item[AV - Ansichten zur Verkehrspolitik] Welche Ziele sollen von der Politik im Verkehrsraum verfolgt werden, welche Probleme sehen die Probanden
\end{description}
Das die Ansichten der Befragten Menschen von demographischen Faktoren abhängen sollte offensichtlich sein.
Drei demographische Faktoren sind für uns besonders interessant: Geschlecht, Alter und Soziale Stellung \footnote{Siehe \cite[33 ff.]{widmer}}.
Außerdem ist entscheidend, ob die Probanden überhaupt in einer Stadt leben.

Dagegen ist nicht völlig offensichtlich, ob das Mobilitätsverhalten der Probanden die Ansichten zur Verkehrspolitik ändern, oder ob eine andere Einstellung der Probanden dafür sorgt, dass andere Verkehrsmittel genutzt werden – Realistisch betrachtet dürfte beides der Fall sein.
Wir beschränken uns allerdings darauf, die Auswirkungen des Mobilititätsverhaltens auf die Ansichten der Probanden zu untersuchen.
So wird ein Fahrradfahrer beispielsweise den Straßenverkehr anders wahrnehmen und unter Umständen auch andere Ansichten zur Verkehrspolitik entwickeln.
Ob dieser Wahrnehmungsunterschied tatsächlich vorhanden ist lässt sich durch möglicherweise durch QD3: \enquote{When travelling within cities, how often do you encounter problems that limit your access to activities, goods or services?} zeigen.

Die übrigen Fragen Zielen auf die Ansichten der Verkehrspolitik und sollen zeigen wie die Ansichten der Probanden mit Demographie und Mobilitätsverhalten zusammenhängen.

\nocite{schulz}

\begin{table}
	\begin{tabularx}{\textwidth} { b{8cm} | r | r | r}
		D10:  Gender.  & abs   & \%    \\ \hline
		male   & 12675 & 45.79 \\
		female & 15005 & 54.51 \\	\hline
		D11:  How old are you?  & abs   & \%    \\ \hline
		15-10  &  1315 &  4.75 \\
		20-30  &  3665 & 13.24 \\
		30-40  &  4372 & 15.79 \\
		40-50  &  4607 & 16.64 \\
		50-60  &  4797 & 17.33 \\
		60-70  &  4701 & 16.98 \\
		70-80  &  3050 & 11.02 \\
		80-90  &  1072 &  3.87 \\
		90-100 &   101 &  0.36 \\ \hline
		D25: Would you say you live in a...? & abs  & \% & \%(Bereinigt) \\ \hline
		Rural area or village  &    9673 & 34.94581 & 34.97 \\
		Small/middle town      &   10650 & 38.47543 & 38.50 \\
		Large town             &    7340 & 26.51734 & 26.53 \\
		NA's                   &      17 &  0.06142 &       \\ \hline
		D61: On the following scale, step '1' corresponds to "the lowest level in the society"; step '10' corresponds to "the highest level in the society". Could you tell me on which step you would place yourself? & abs  & \% & \%(Bereinigt) \\ \hline
		1,2  & 1109 &  4.007  &       4.099 \\
		3,4  & 5841 &  21.102 &       21.588 \\
		5,6  & 13361&  48.270 &       49.381 \\
		7,8  & 6207 &  22.424 &       22.940 \\
		9,10 & 539  &  1.947  &       1.992 \\
		NA's & 623  &  2.251 \\
	\end{tabularx}
	\caption{Variablenblock Demographie mit Verteilungen}
\end{table}

\begin{table}
	\begin{tabularx}{\textwidth} { b{12cm} | r | r}
QD1: How often do you...? \\ \hline
QD1\_1: Use a car (whether as a driver or a passenger) & abs & \% \\ \hline
Several times a day          &   8749 & 31.6077 \\
Once a day                   &   4139 & 14.9530 \\
Two or three times a week    &   5267 & 19.0282 \\
About once a week            &   2208 &  7.9769 \\
Two or three times a month   &   1559 &  5.6322 \\
Less often                   &   2463 &  8.8981 \\
Never                        &   3230 & 11.6691 \\
DK                           &     65 &  0.2348 \\ \hline
QD1\_2: Use public transports & abs & \% \\ \hline

Several times a day          &   2546 &  9.1980 \\
Once a day                   &   1576 &  5.6936 \\
Two or three times a week    &   3203 & 11.5715 \\
About once a week            &   1990 &  7.1893 \\
Two or three times a month   &   2697 &  9.7435 \\
Less often                   &   7902 & 28.5477 \\
Never                        &   7699 & 27.8143 \\
DK                           &     67 &  0.2421 \\ \hline
QD1\_3: Ride a motorbike (whether as a driver or a passenger) & abs & \% \\ \hline

Several times a day          &    318 &  1.1488 \\
Once a day                   &    249 &  0.8996 \\
Two or three times a week    &    408 &  1.4740 \\
About once a week            &    285 &  1.0296 \\
Two or three times a month   &    337 &  1.2175 \\
Less often                   &   1426 &  5.1517 \\
Never                        &  24510 & 88.5477 \\
DK                           &    147 &  0.5311 \\ \hline
QD1\_4: Cycle & abs & \% \\ \hline

Several times a day          &   2059 &  7.4386 \\
Once a day                   &   1567 &  5.6611 \\
Two or three times a week    &   3031 & 10.9501 \\
About once a week            &   1726 &  6.2355 \\
Two or three times a month   &   1666 &  6.0188 \\
Less often                   &   3976 & 14.3642 \\
Never                        &  13535 & 48.8981 \\
DK                           &    120 &  0.4335 \\
	\end{tabularx}
	\caption{Variblenblock Mobilitätsverhalten mit Verteilungen}
\end{table}


\begin{table}
	\begin{tabularx}{\textwidth} { b{8cm} | r | r | r}
		QD3: When travelling within cities, how often do you encounter problems that limit your access to
activities, goods or services? & avg & \% & \% (Bereinigt) \\ \hline
		Often      &    2345 &  8.472  &       8.843  \\
		Sometimes  &    7097 & 25.639  &      26.763  \\
		Rarely     &    8998 & 32.507  &      33.932  \\
		Never      &    7643 & 27.612  &      28.822  \\
		DK         &     435 &  1.572  &       1.640  \\
		NA's       &    1162 &  4.198  \\ \hline
		QD4: Do you think that the following issues are an important problem or not within cities? \\ \hline
		QD4\_1: Road congestion & avg & \% \\ \hline 
		A very important problem        &     7772 & 28.078 \\
A fairly important problem          &    11524 & 41.633 \\
Not a very important problem        &     5845 & 21.116 \\
Not an important problem at all     &     2064 &  7.457 \\
DK                                  &      475 &  1.716 \\ \hline
		QD4\_2: Noise pollution & avg & \% \\ \hline 
		A very important problem        &     7180 & 25.939 \\
A fairly important problem          &    11436 & 41.315 \\
Not a very important problem        &     6591 & 23.811 \\
Not an important problem at all     &     2013 &  7.272 \\
DK                                  &      460 &  1.662 \\ \hline
		QD4\_3: Air pollution & avg & \% \\ \hline 

A very important problem            &    10342 & 37.363 \\
A fairly important problem          &    10912 & 39.422 \\
Not a very important problem        &     4462 & 16.120 \\
Not an important problem at all     &     1606 &  5.802 \\
DK                                  &      358 &  1.293 \\

	\end{tabularx}
	\caption{Variblenblock Ansichten zur Mobilität und Mobilitätswahrnehmung}
\end{table}

% subsection Metadaten (end)
