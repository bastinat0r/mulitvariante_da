\documentclass[a4paper,12pt,titlepage]{scrartcl}

%Pakete
%\usepackage[left=3cm,right=2cm,top=2cm,bottom=3cm]{geometry} %Seitenränder
\usepackage[utf8]{inputenc} % ermöglicht die direkte Eingabe der Umlaute 
\usepackage [german]{babel} %Spracheinstellungen

\usepackage {csquotes} % Anführungszeichen nach dem Stil \enquote{ich bin zitiert.}
\usepackage{subscript} % erlaubt Tiefstellen von Zahlen und Text
\usepackage{tabularx} %krassere tabellen
\sloppy %macht ungefähren Blocksatz, wenn nichts anderes an Trennhilfen was nützt

% Hurenkinder und Schusterjungen verhindern
\clubpenalty10000
\widowpenalty10000
\displaywidowpenalty=10000

\usepackage[final]{graphicx}
\usepackage{floatflt}

\usepackage [
citestyle = authoryear,
bibstyle = philosophy-classic,% Zitierstil
isbn=false,                % ISBN nicht anzeigen, %gleiches geht mit nahezu allen anderen Feldern
doi=false,
pagetracker=true,          % ebd. bei wiederholten ngaben (false=ausgeschaltet, page=Seite, spread=Doppelseite, true=automatisch)
%ümaxbibnames=3,            % maximale Namen, die im Literaturverzeichnis angezeigt werden 
maxcitenames=3,            % maximale Namen, die im Text angezeigt werden, ab 4 wird u.a. nach den ersten Autor angezeigt
autocite=inline,           % regelt Aussehen für \autocite (inline=\parancite)
block=space,               % kleiner horizontaler Platz zwischen den Feldern
backref=false,              % Seiten anzeigen, auf denen die Referenz vorkommt
backrefstyle=three+,       % fasst Seiten zusammen, z.B. S. 2f, 6ff, 7-10
date=short                % Datumsformat
]{biblatex}
\newcommand{\todo}[1]{\marginpar{#1}}
\bibliography{~/sources}

\DeclareBibliographyDriver{report}{%
  \usebibmacro{bibindex}%
  \usebibmacro{begentry}%
  \usebibmacro{author}%
  \setunit{\labelnamepunct}\newblock
  \usebibmacro{title}%
  \newunit
  \printlist{language}%
  \newunit\newblock
  \usebibmacro{byauthor}%
  \newunit\newblock
  \printfield{type}%
  \setunit*{\addspace}%
  \printfield{number}%
  \newunit\newblock
  \printfield{version}%
  \newunit
  \printfield{note}%
  \newunit\newblock
  \usebibmacro{institution+location+date}%
  \newunit\newblock
  \usebibmacro{chapter+pages}%
  \newunit
  \printfield{pagetotal}%
  \newunit\newblock
  \iftoggle{bbx:isbn}
    {\printfield{isrn}}
    {}%
  \newunit\newblock
  \usebibmacro{doi+eprint+url}%
  \newunit\newblock
 \usebibmacro{addendum+pubstate}%
  \setunit{\bibpagerefpunct}\newblock
  \usebibmacro{pageref}%
  \usebibmacro{finentry}}
\usepackage[colorlinks=true,linkcolor=black,citecolor=black,urlcolor=black,breaklinks=true]{hyperref}
%\usepackage[style=authortitle-icomp]{biblatex}

%\bibliography{Pfad/zur/Bibliographie-Datei/Dateiname} 


\title{Urbane Mobiltität in Europa}
\subtitle{1. Forschungspapier – Multivariante Datenanalyse}
\author{Sebastian Mai}
\date{\today} % sollte man u.U. anpassen ;)


\begin{document}
\maketitle
\tableofcontents
\listoftables
\section{Relevanz}
\subsection{Mobilität im Urbanen Raum} % (fold)
	
	Mobilität ist ein wichtiger Faktor für alle Menschen in einer Modernen Gesellschaft.
	Wir möchten untersuchen, wie sich die Ansichten der Menschen in Städten zu ihrem Mobilitätsverhalten in Beziehung setzen lassen.

\begin{quote}
	Transport is a key enabler of social and economic development, and the transport sector
accounts for 9 million jobs across the EU. As 68\% of EU citizens live in urban areas 1 ,
urban transport is particularly important to future growth.
\emph{\parencite[2]{ebs406en}}
\end{quote}

% subsection Mobilität im Urbanen Raum (end)

% erhebungsmethoden und foo
% 1 UV: Demographie
% 2 UV: Mobilitätsverhalten
% 3 AV: Ansichten zur Mobilität

\section{Theoretische Vorüberlegung}

\subsection{Begriffe}
\subsubsection*{Mobilität}

Mobilität ist die Möglichkeit zum Wechsel von Orten, mit der auch Symbolische und Soziale Aspekte verknüpft sind Verkehr lässt sich als realisierte Mobilität betrachten \parencite[6]{fh7}.
Bestimmend für den Möglichkeitsraum der Mobilität sind die Verkehrsmittel die wir benutzen – also die verfügbaretechnischen Mittel die uns von A nach B befördern und unser Mobilitätsverhalten prägen.
Unterscheiden kann man zwischen Physicher- und Sozialer Mobilität, also der Möglichkeit sich zwischen Verschiedenen Sozialen Räumen zu bewegen \parencite[67 f.]{schulz}.

\subsubsection*{Urbanität}
Urbanität findet sich sowohl im Sozialen- als auch im Räumlichen gefüge der Gesellschaft wieder.
Entscheidend für die Mobilität ist wohl, das in Urbanen Umgebungen wesentlich kürzere Wege bewältigt werden müssen als außerhalb der Städte.
Im Gegenzug dazu ist der Verkehr in Städten wesentlich dichter und die Auslastung der Verkehrsmittel ist wesentlich höher als in ländlichen Regionen.

\subsection{Makroebene}
Die Theorien der Makroebene beschäftigen sich vor allem mit den Auswirkungen von Mobilität auf die Gesellschaft \parencite[6 ff.]{widmer}.
Ein Beispiel dafür ist die Beschleunigung des Lebenstempos \parencite{rosa,rosa1} oder die Flexibilisierung der Gesellschaft \parencite{sennett}.

Eine Entwicklung der letzten Jahre ist, dass die Verbindung zwischen Physicher- und Sozialer Mobilität zunehmend abnimmt, da Kommunikation auch über große Distanzen stattfinden kann (\enquote{virtual mobility}) – zusammen mit den gestiegenen Möglichkeiten der Logistik bedeutet das, das obwohl unsere Mobilität – gedacht als die Möglichkeit Dinge zu erreichen zunimmt, die Tatsächlich zurückgelegten Wege aber kürzer werden.
Damit bietet sich natürlich wiederum die Möglichkeit mehr Dinge zu erreichen, oder die selben Dinge mit anderen Verkehrsmitteln.
Außerdem wird Mobilität nicht mehr nur als Möglichkeit, sondern als Realität wahrgenommen \parencite[17 f.]{widmer}.

\subsection{Mikroebene}
Mobilität hat nicht zuletzt auch Auswirkungen auf jeden einzelnen Akteur und ist sowohl in seiner Eigenschaft als Möglichkeitsraum (vgl \cite{riessman}) als auch in der Symbolischen Bedeutung, die unseren Verkehrsmitteln innewohnt, identitätsstiftend.
Insofern ist auch davon auszugehen, das unser Mobilitätsverhalten unsere Ansichten zur Mobilität beeinflussen.


\subsection{Mesoebene}

Die meisten Theorien zur Mobilität verknüpfen Handlungsmotivationen in der Mikroebene mit Auswirkungen sowohl in Mikro- als auch Makroebene, diese Theorien lassen sich in zwei Gruppen einteilen: \enquote{Rational choice theorie} (Theorie der Rationalen Entscheidung) und \enquote{cultural or life-style theories} \parencite{widmer}.

\subsubsection*{Rational Choice}
Kern der rational choice theorie ist die Annahme, das alle Akteure rationale Entscheidungen im Sinne einer Kosten-Nutzen-Maximierung für sich selbst treffen \parencite[][19]{wikiRationalChoice, widmer}.
Das einfachste Modell dieser Kathegorie ist vermutlich der Homo Oeconomicus.
Neuere RC-Theorien gehen davon aus, dass die getroffenen Entscheidungen unter bestimmten Beschränkungen (bounded rationality) und ohne vollständiges Wissen über deren Auswirkungen getroffen werden.
Die Theorien haben das Problem, dass die subjektiven Kosten und der subjektive Nutzen einer Handlung durch einen Beobachter praktisch nicht abschätzbar sind und altruisitsche Handlungsmotive ausgeschlossen werden.

Da die kosten für die meisten Handlungen im bereich der Mobilität allerdinsgs relativ hoch sind spielt die Abschätzung von Kosten und Nutzen beim Treffen einer Entscheidung relativ sicher eine Rolle, und die Motivation eine Informierte Entscheidung zu treffen ist relativ hoch \parencite[21]{widmer}\footnote{In der Qulle beziehen sich die beiden Autoren explizit auf Arbeitsmobilität, ich denke aber die Aussage lässt sich durchaus verallgemeinern.}.

\subsubsection*{Mobilitätskultur}
\begin{quote}
	\dots aspects of mobility culture seem clearly detectable. In order to become mobile it is
important to know about traffic signs and speed limits, to know about train schedules and how
to research them, to know how much luggage is allowed on an airplane etc. \emph{\parencite[24]{widmer}}
\end{quote}

Ein anderer Ansatz als der der Rationlen Entscheidungen ist der Bezug auf eine \emph{Mobilitätskultur} \parencite[22 f.][6 ff.]{widmer, fh7}.
Das Mobilitätsverhalten eines Einzelnen ist damit Ergebnis eines Sozialisierungsprozesses.
Dabei können nicht nur Unterschiede in der Mobilitätskultur aus verschiedenen Kulturkreisen verglichen werden, sondern durchaus auch innerhalb eines Gebietes mit sehr ähnlichen Kulturen z.B. innerhalb Europas – so lassen sich zwischen Frankfurt und Zürich deutliche Unterschiede in der Mobilitätskultur feststellen \parencite{fh7}.

Auch verschiedene Randbedingungen bei ähnlicher Kultur kann zu unterschieden führen \parencite[24]{widmer}.
Dabei sind auch Subkulturen und die Mischung an verschiedenen Subkulturen an denen ein Einzelner teilnimmt entscheidend für das Mobilitätsverhalten.
So sind die Einstellungen zum ÖPNV in verschiedenen Städten nicht nur durch rationales Denken, sondern auch in der Kultur der jeweiligen Stadt verankert.

Diese Einstellungen sind Bestandteile eines Life-Style \parencite[24 f.]{widmer} der wiederum als Identitätsangebot fungiert.
Die kulturellen Gegebenheiten schlagen sich auch in Rollenbildern und Erwartungen an diese wieder.
Den bestehenden Leitbildern wird teilweise auch völlig unreflektiert gefolgt \parencite[27]{widmer}.
Das theoretische Fundament dieser Ansätze findet sich unter anderem im Symbolischen Interaktionismus \parencite[27][14 ff.]{widmer, schulz}.

\section{Datensatz}
% Metadata für unseren Datensatz, 3-4 seiten
% mit univariaten verteilungen als tabelle

\nocite{eb794}
\subsection{Erhebung}

Für den Datensatz wurden Daten aus 28 Verschiedenen Ländern der EU erhoben, die Erhebung fand statt zwischen 25.04.2013 und 09.06.2013.

In jedem Land wurden ca. 1000 Menschen in computergeleiteten face-to-face Interviewes befragt, in Deutschland wurden 1000 Menschen in West- und 500 in Ostdeutschland befragt. In Luxembourg aufgrund der geringen Bevölkerungszahl nur 600 und in Großbritannien 1300. Insgesamt wurden 27680 Menschen befragt \parencite{eceuropaeu}.

\subsection{Variablen}

Bei den Ansichten zur Verkehrspolitik gibt es vermutlich zwei Konkurrierende Ziele: Einerseits die Erhaltung und Erweiterung der eigenen Mobilität und andererseits die Förderung der Nachhaltigkeit der genutzten Verkehrsmittel\footnote{Hier konkuriert ein kultureller Wert mit dem Egoismus des Homo Oeconomicus.}.

Für die Auswertung der Umfragen verwenden wir drei Blöcke von Variablen:
\begin{description}
	\item[UV - Demographie] Demographische Faktoren, wie Alter, sozialer Status und Geschlecht, Nationalität
	\item[UV - Mobilitätsverhalten] Mit welchen Mitteln bewegen sich die Probanden im Straßenverkehr
	\item[AV - Ansichten zur Verkehrspolitik] Welche Ziele sollen von der Politik im Verkehrsraum verfolgt werden, welche Probleme sehen die Probanden
\end{description}
Das die Ansichten der Befragten Menschen von demographischen Faktoren abhängen sollte offensichtlich sein.
Drei demographische Faktoren sind für uns besonders interessant: Geschlecht, Alter und Soziale Stellung \footnote{Siehe \cite[33 ff.]{widmer}}.
Außerdem ist entscheidend, ob die Probanden überhaupt in einer Stadt leben – welche Datensätze interessant sind wird also maßgeblich durch QD25 bestimmt.

Dagegen ist nicht völlig offensichtlich, ob das Mobilitätsverhalten der Probanden die Ansichten zur Verkehrspolitik ändern, oder ob eine andere Einstellung der Probanden dafür sorgt, dass andere Verkehrsmittel genutzt werden – Realistisch betrachtet dürfte beides der Fall sein.
Wir beschränken uns allerdings darauf, die Auswirkungen des Mobilititätsverhaltens auf die Ansichten der Probanden zu untersuchen.
So wird ein Fahrradfahrer beispielsweise den Straßenverkehr anders wahrnehmen und unter Umständen auch andere Ansichten zur Verkehrspolitik entwickeln.

Ob dieser Wahrnehmungsunterschied tatsächlich vorhanden ist lässt sich durch möglicherweise durch QD3: \enquote{When travelling within cities, how often do you encounter problems that limit your access to activities, goods or services?} zeigen.
Die übrigen Fragen Zielen auf die Ansichten der Verkehrspolitik und sollen zeigen wie die Ansichten der Probanden mit Demographie und Mobilitätsverhalten zusammenhängen.
\\

\nocite{schulz}

\begin{table}[h]
	\begin{tabularx}{\textwidth} { b{8cm} | r | r | r}
		D10:  Gender.  & abs   & \%    \\ \hline
		male   & 12675 & 45.79 \\
		female & 15005 & 54.51 \\	\hline
		D11:  How old are you?  & abs   & \%    \\ \hline
		15-10  &  1315 &  4.75 \\
		20-30  &  3665 & 13.24 \\
		30-40  &  4372 & 15.79 \\
		40-50  &  4607 & 16.64 \\
		50-60  &  4797 & 17.33 \\
		60-70  &  4701 & 16.98 \\
		70-80  &  3050 & 11.02 \\
		80-90  &  1072 &  3.87 \\
		90-100 &   101 &  0.36 \\ \hline
		D25: Would you say you live in a...? & abs  & \% & \%(Bereinigt) \\ \hline
		Rural area or village  &    9673 & 34.94581 & 34.97 \\
		Small/middle town      &   10650 & 38.47543 & 38.50 \\
		Large town             &    7340 & 26.51734 & 26.53 \\
		NA's                   &      17 &  0.06142 &       \\ \hline
		D61: On the following scale, step '1' corresponds to "the lowest level in the society"; step '10' corresponds to "the highest level in the society". Could you tell me on which step you would place yourself? & abs  & \% & \%(Bereinigt) \\ \hline
		1,2  & 1109 &  4.007  &       4.099 \\
		3,4  & 5841 &  21.102 &       21.588 \\
		5,6  & 13361&  48.270 &       49.381 \\
		7,8  & 6207 &  22.424 &       22.940 \\
		9,10 & 539  &  1.947  &       1.992 \\
		NA's & 623  &  2.251 \\
	\end{tabularx}
	\caption{Variablenblock Demographie mit Verteilungen}
\end{table}

\begin{table}[h]
	\begin{tabularx}{\textwidth} { b{12cm} | r | r}
QD1: How often do you...? \\ \hline
QD1\_1: Use a car (whether as a driver or a passenger) & abs & \% \\ \hline
Several times a day          &   8749 & 31.6077 \\
Once a day                   &   4139 & 14.9530 \\
Two or three times a week    &   5267 & 19.0282 \\
About once a week            &   2208 &  7.9769 \\
Two or three times a month   &   1559 &  5.6322 \\
Less often                   &   2463 &  8.8981 \\
Never                        &   3230 & 11.6691 \\
DK                           &     65 &  0.2348 \\ \hline
QD1\_2: Use public transports & abs & \% \\ \hline

Several times a day          &   2546 &  9.1980 \\
Once a day                   &   1576 &  5.6936 \\
Two or three times a week    &   3203 & 11.5715 \\
About once a week            &   1990 &  7.1893 \\
Two or three times a month   &   2697 &  9.7435 \\
Less often                   &   7902 & 28.5477 \\
Never                        &   7699 & 27.8143 \\
DK                           &     67 &  0.2421 \\ \hline
QD1\_3: Ride a motorbike (whether as a driver or a passenger) & abs & \% \\ \hline

Several times a day          &    318 &  1.1488 \\
Once a day                   &    249 &  0.8996 \\
Two or three times a week    &    408 &  1.4740 \\
About once a week            &    285 &  1.0296 \\
Two or three times a month   &    337 &  1.2175 \\
Less often                   &   1426 &  5.1517 \\
Never                        &  24510 & 88.5477 \\
DK                           &    147 &  0.5311 \\ \hline
QD1\_4: Cycle & abs & \% \\ \hline

Several times a day          &   2059 &  7.4386 \\
Once a day                   &   1567 &  5.6611 \\
Two or three times a week    &   3031 & 10.9501 \\
About once a week            &   1726 &  6.2355 \\
Two or three times a month   &   1666 &  6.0188 \\
Less often                   &   3976 & 14.3642 \\
Never                        &  13535 & 48.8981 \\
DK                           &    120 &  0.4335 \\
	\end{tabularx}
	\caption{Variblenblock Mobilitätsverhalten mit Verteilungen}
\end{table}


\begin{table}[h]
	\begin{tabularx}{\textwidth} { b{8cm} | r | r | r}
		QD3: When travelling within cities, how often do you encounter problems that limit your access to
activities, goods or services? & abs & \% & \% (Bereinigt) \\ \hline
		Often      &    2345 &  8.472  &       8.843  \\
		Sometimes  &    7097 & 25.639  &      26.763  \\
		Rarely     &    8998 & 32.507  &      33.932  \\
		Never      &    7643 & 27.612  &      28.822  \\
		DK         &     435 &  1.572  &       1.640  \\
		NA's       &    1162 &  4.198  \\ \hline
		QD4: Do you think that the following issues are an important problem or not within cities? \\ \hline
		QD4\_1: Road congestion & abs & \% \\ \hline 
		A very important problem        &     7772 & 28.078 \\
A fairly important problem          &    11524 & 41.633 \\
Not a very important problem        &     5845 & 21.116 \\
Not an important problem at all     &     2064 &  7.457 \\
DK                                  &      475 &  1.716 \\ \hline
		QD4\_2: Noise pollution & abs & \% \\ \hline 
		A very important problem        &     7180 & 25.939 \\
A fairly important problem          &    11436 & 41.315 \\
Not a very important problem        &     6591 & 23.811 \\
Not an important problem at all     &     2013 &  7.272 \\
DK                                  &      460 &  1.662 \\ \hline
		QD4\_3: Air pollution & abs & \% \\ \hline 

A very important problem            &    10342 & 37.363 \\
A fairly important problem          &    10912 & 39.422 \\
Not a very important problem        &     4462 & 16.120 \\
Not an important problem at all     &     1606 &  5.802 \\
DK                                  &      358 &  1.293 \\

	\end{tabularx}
	\caption{Variblenblock Ansichten zur Mobilität und Mobilitätswahrnehmung}
\end{table}
\begin{table} {
		\begin{tabular}{l | l | l}
			von & bis & Land \\ \hline
			25.05.2013 & 09.06.2013 & Belgium \\
			25.05.2013 & 02.06.2013 & Bulgaria \\
			24.05.2013 & 06.06.2013 & Czech Republic \\
			25.05.2013 & 09.06.2013 & Denmark \\
			24.05.2013 & 09.06.2013 & Germany \\
			24.05.2013 & 09.06.2013 & Estonia \\
			25.05.2013 & 09.06.2013 & Ireland \\
			25.05.2013 & 08.06.2013 & Greece \\
			24.05.2013 & 09.06.2013 & Greece \\
			24.05.2013 & 08.06.2013 & France \\
			25.05.2013 & 07.06.2013 & Italy \\
			24.05.2013 & 09.06.2013 & Rep. of Cyprus \\
			25.05.2013 & 09.06.2013 & Latvia \\
			25.05.2013 & 09.06.2013 & Lithuania \\
			25.05.2013 & 09.06.2013 & Luxembourg \\
			25.05.2013 & 09.06.2013 & Hungary \\
			24.05.2013 & 09.06.2013 & Malta \\
			24.05.2013 & 09.06.2013 & Netherlands \\
			24.05.2013 & 09.06.2013 & Austria \\
			25.05.2013 & 09.06.2013 & Poland \\
			28.05.2013 & 09.06.2013 & Portugal \\
			25.05.2013 & 04.06.2013 & Romania \\
			25.05.2013 & 09.06.2013 & Slovenia \\
			25.05.2013 & 09.06.2013 & Slovakia \\
			25.05.2013 & 09.06.2013 & Finland \\
			25.05.2013 & 09.06.2013 & Sweden \\
			25.05.2013 & 09.06.2013 & United Kingdom \\
			25.05.2013 & 09.06.2013 & Croatia \\
		\end{tabular}}
	\caption{Erhebungszeiträume und Länder: \cite{eb794}}
\end{table}

% subsection Metadaten (end)

\newpage
\sloppy
\printbibliography
\end{document}
